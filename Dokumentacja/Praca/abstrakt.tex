\documentclass[12pt, letterpaper]{article}
\usepackage[utf8]{inputenc}
\usepackage[T1]{fontenc}
\pagenumbering{gobble}

\begin{document}

\begin{center}
Politechnika Śląska

Wydział Automatyki, Elektroniki i Informatyki\break

\textbf{Abstrakt pracy dyplomowej magisterskiej pt.}

,,Predykcja zachowań zawodników podczas meczu piłki siatkowej''
\end{center}
autor: inż. Przemysław Gepfert\\
promotor: dr hab. prof. Pol. Śl. Krzysztof Simiński\\\\

Celem pracy jest przetwarzanie i analiza danych opisujących zagrania poszczególnych zawodników podczas meczów piłki siatkowej. Problem polega na wykorzystaniu metod eksploracji danych z meczów siatkarskich w celu znalezienie zależności, trudno zauważalnych dla człowieka, w grze poszczególnych zawodników. Otrzymane wyniki mają za zadanie pomóc trenerom piłki siatkowej przewidywać najbardziej prawdopodobne zagrania zawodników, tak by zwiększyć szansę na zdobycie kolejnych punktów w meczu i co za tym idzie również zwiększyć szansę na osiągnięcie końcowego sukcesu w postaci zwycięstwa własnej drużyny w meczu. 

Efektem pracy są badania przeprowadzone na rzeczywistych danych, pochodzących z meczów sezonu 2019/2020, rozegranych w profesjonalnej lidze siatkarskiej. 
Badania te objęły sprawdzene istotności atrybutów dla trzech podstawowych zagrań w siatkówce: przyjęcia, rozegrania oraz ataku, a także porównanie wyników uzyskiwanych dla klasyfikacji kierunku rozegrania. Do przeprowadzenia klasyfikacji wykorzystano metody budowy drzew decyzyjnych oraz odkrywania reguł asocjacyjnych.

W celu przeprowadzenia badań stworzona została aplikacja desktopowa. Pozwala ona wczytywać do bazy danych pliki o rozszerzeniu ,,dvw'', zawierające zapisy każdego odbicia piłki w meczu, a następnie na podstawie wczytanych danych przeprowadzić analizę, wybierając odpowiedniego zawodnika, rodzaj zagrania, metodę analizy oraz parametry analizy. Dzięki współpracy z profesjonalnym statystykiem siatkarskim aplikacja ta została zrealizowana tak, by móc być w przyszłości wykorzystana przez sztaby trenerskie.

Na podstawie uzyskanych wyników badań można wnioskować, że wykorzystanie metod eksploracji danych jak najbardziej pozwala w szybki sposób uzyskać bardzo wartościową analizę gry zawodników, która dzięki wyszukiwaniu zależności często pomijanych do tej pory przez sztaby
trenerskie, może pomóc drużynom w uzyskiwaniu lepszych wyników sportowych.

\end{document}